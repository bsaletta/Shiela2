\documentclass{report}
\usepackage{fullpage}
\usepackage{graphicx}
\graphicspath{{figures/}}
\title{Design of An Innovative, Highly Maneuverable, Stealthy Unmanned Underwater Vehicle with ISR Capabilities }
\setlength{\parindent}{4em}
\author{Andrew Blancarte\\ Ketton James\\ Brian Martin\\ Abraham Paucar\\ Ben Saletta}
\date{\today}
\renewcommand{\chaptername}{Section}
\begin{document}
\maketitle
\tableofcontents
\listoffigures
\chapter{Phase I: Propulsor Development, SHEILA-D}
\section{Overview}
SHEILA-D stands for Subaquatic Hydrodynamicaly propelled Explorer Implementation Los Angeles - Demonstrator. The driving force behind the development of this unmanned underwater vehicle is the innovative propulsion system. This system was the foundation of the first phase of development, a proof of concept for this screw based propulsion system. As this phase was a deliverable for a Machine Design Class there is a detailed overview of this phase included in appendix A. 
\section{Constraints}
As this initial design was part of a class project there were many constraints that came along with it. The primary constraint was time. The project was completed, from design to testing, in 35 days to meet the deadline imposed by the academic calender. There were also significant cost constraints because this phase was funded completely out of the personal funds of the members involved. The constraints eliminated several custom manufacturing techniques due to extensive lead times and costs. This led to the final construction consisting mainly of parts shaped from hardware store components.
\section{Design}
The initial design phase took the first 5 days in which the foundational fluid mechanics, the drive train and power calculations were completed in these days. Further redesign was included in with the manufacturing process when the aforementioned constraints restricted the construction of the device as designed.
\subsection{Fluids Calculations}
The fluids calculations were an important part of the initial design because of the roll they played in moving the device through the water. To complete these calculations the device was broken into two sections. First the cylinder holding the screw and second the cone used to accelerate the fluid. For the cylinder an assumption was made that a constant force would be imparted to the fluid along the surface of the screw blades. This constant force would act on the fluid mass between the vanes therefore, assuming a constant density, the vanes would provide a constant acceleration along this section. With a known length of the cylinder and this calculated acceleration the speed of the fluid exiting the cylinder was calculated. Conservation of momentum was then applied to the fluid as it passed through the final cone and nozzle giving the final thrust force of the propulsion system. These calculations were plugged into an Excel spreadsheet and various geometries were iterated through to find the optimal configuration. A more detailed walk through of the fluids analysis is included in appendix A.
\subsection{Drive Train}
The fluids calculations provided a value for the required torque and angular velocity. The drive train was designed to meet both of these criterion. The major difficulty with the drive train in this design was transmitting the torque from the motors to the external shell. To solve this problem and provide sufficient torque two motors were fitted with gearboxes then mated to the sun gear in a planetary gear set that transmitted the torque out to the rotating cylinder as seen in Figure 1.1. The planetary gear system was 3D printed to minimize cost and manufacturing time.
\begin{figure}[h]
\centering
\includegraphics[width=15cm]{"Section View"}
\caption{Drive Train Design Section View}
\end{figure}
\subsection{Power Supply}
These motors required a significant amount of current at a constant 12 volts. These requirements went into the power design calculations and searches for sufficient batteries. A single 12 volt lithium battery was selected due to its high current discharge, compact form, and high capacity characteristics. The design of the power supply was integrated into the design of the drive train due to the important relationship between the easily available motors and batteries for a reasonable operational time. The power supply had to fit between the two motors and is shown as a black box in Figure 1.1.
\section{Construction}
The construction of this device was completed in a very short time span using commercial off the shelf components. This involved PVC pipes of various diameters, cutting and thermoforming sheets of acrylic, and fastening components with traditional fasteners as well as epoxy. As the construction proceeded difficulties were discovered and the design was adjusted to compensate for these problems. One of the major difficulties was the manufacture of the helix screw. This was solved by thermoforming cut acrylic then filling gaps created in the vanes with waterproof tape. Other adjustments and quick fixes were applied through out the process. The construction was a challenge to complete within the time frame at an acceptably low cost.  
\section{Testing}
Once construction was completed the device needed to be tested to confirm that the propulsion system did indeed propel the device through the water. To do the testing properly the vehicle was taken to the Puddingstone reservoir at dawn and ran along the deck of a boat dock. It took a few tests to get the device operational and, while the vehicle could not be set off on its own at this stage, it pumped a significant amount of water though the body and out the cone and nozzle assembly. Many things were learned about the design during the testing session and made note of for future design. The major lessons were that the assumptions that were made in the fluid calculations were wrong and that without control surfaces the vehicle would reach an equilibrium rotation state where the outermost cylinder would rotate and counter the rotation of the center cylinder. 
\chapter{Phase II: Summer Research and Testing}
During the summer of 2014 the group members of Team UV did not entirely part ways for three months. The group decided that it would be in the best interest of the group and design itself to meet twice a month where each member would give presentation on standard material along with research they have found during the two week period between meetings. These meetings started in June of 2014 and ended when the 2014 Fall period started, September 2014.  Each of the meetings held were roughly five to seven hours long. The format of the presentations were standardized however each member provided their own research and information. Presentation format can be found in appendix B. In the summer of 2014 Team UV spent roughly 240 hours as a group towards the senior project itself. That number includes research done outside of the meeting along with the meetings themselves. The PowerPoint presentations contained three sections.
\section{Open Mind}
The group was given a situation/problem that has or could happen in the world we live in today and each member had to give insight or propose a solution to the problem/situation. Each member were asked to provide their solution along with reasoning  as to why their solution would be efficient. Since Team UV are composed of mechanical engineering undergraduates, the solutions based on the prompts were required to use some aspect of engineering. For example, one open mind prompt addressed the issue that many people in developing countries lack the access to power. The group was asked if they were tasked with developing an inexpensive “do it yourself” power generation system for use in a 3rd world country, what kind of engineering considerations might you take into account. The idea behind the open mind prompt was to keep the group members mind engaged during the break and keep that engineering mentality sparked. Also, this prompt was to have the members think outside the box and to aid in developing creative/innovative ideas for use in the senior project design.
\section{Well-Read}
Members were asked to look into articles/publications and present something they found to be interesting, innovative, advancements, etc. The articles were not limited to strictly the engineering field however it was recommended.  Articles presented from the group ranged from next generation wind turbines to theory behind the process of suction eating of fish. The idea of the well-read was to give insight to the group of remarkable advancements in science and technology throughout the world.
\section{Presentation}
The presentation was an open ended section which allowed for each member to present some research they did during the two week period between meetings. The research was required to aid in the design of the senior project in areas such as theory, analysis, design, controls, manufacturing, and/or testing. Research in CLT propellers, computational fluid dynamics, Arduino coding, wake turbulence, underwater designs and coatings, power management, component selection for marine research, smart ducts, submarine design/research, and many others. 
\section{Summer Research Summaries}
\subsection{Shark Skin}
\begin{figure}[h]
\centering
\includegraphics[width=5cm]{"Shark Skin"}
\caption{Close up of Shark Skin}
\end{figure}
Sharks move very efficiently thanks to the characteristics of their skin.  Anything but smooth, shark skin has individual scales, called dermal denticles, with narrow passages that reduce friction drag and accelerate the flow along its length.  These scales also flex and realign to reduce biofouling which can negatively affect flow over the body.

\subsection{Vantablack}
\begin{figure}[h]
\centering
\includegraphics[width=5cm]{"Vantablack"}
\caption{Vantablack on Aluminum Substrate}
\end{figure}
Surrey NanoSystems has engineered a new super-black material called Vantablack.  This material composed of “Vertically Aligned Carbon Nanotube Arrays” absorbs 99.965\% of incident light.  Photons are allowed into the material and are then blocked and trapped from leaving.  This material is already in high demand for space and stealth applications. 

\subsection{Cephalopod Skin}
\begin{figure}[h]
\centering
\includegraphics[width=5cm]{"Cephalopod Skin"}
\caption{Cephalopod Skin}
\end{figure}
Two teams of researchers from Rice University and MIT were tasked with developing material which could replicate the camouflaging abilities of cephalopods.  The class of mollusks which include squid, possess skin that can manipulate its own color and texture.  Rice University engineering a rigid aluminum nanorod display panel which displays an intense color spectrum.  MIT, on the other hand, created a flexible display that can change color and texture but with limited color spectrum.

\subsection{Waterproofing Sensors}
\begin{figure}[h]
\centering
\includegraphics[width=4cm]{"Potting"}
\includegraphics[width=6cm]{"Conformal Coating"}
\caption{Potting and Conformal Coating}
\end{figure}
Using sensors underwater with a strict budget takes a bit of creativity.  Two popular methods used by hobbyists and professionals alike are: Potting and Conformal Coating.  Potting is the method of filling an electronic assembly with a solid or gelatinous compound such as a thermosetting plastic or silicone.  This blocks water and increases shock resistance.  Conformal Coating offers the same benefits of potting but is lighter and non-permanent which allows for reworking.
\subsection{Arduino}
\begin{figure}[h]
\centering
\includegraphics[width=5cm]{"Arduino UNO"}
\caption{Arduino Uno Microprocessor}
\end{figure}
Microprocessing is made accessible to all thanks to companies like Arduino.  Boards housing everything from input and output terminals to a microprocessor can link the electrical world to the physical world.  Users start out controlling the lighting sequencing of LED’s to eventually controlling more advanced mechanical systems.  Topics presented to the team include LED control, reading data from sensors, and ways to expand the factory limitations of Arduino boards.  
\subsection{CFD and Python}
Computational Fluid Dynamics, CFD, is a powerful tool for the analysis of fluid behavior. The computational techniques used to break down the navier-stokes equations have applications that stretch beyond the realm of fluid dynamics and allow the modeling of many non linear mathematical phenomena within a computational environment. This presentation was an introduction to the ideas behind CFD and the way that an analysis could be written using the Python coding language.
\subsection{Physical Understanding of Fluids Equations and their Application to CFD}
To apply any of the fundamental fluids equations to a device a physical understanding of the meaning behind the mathematics is crucial. This presentation addressed four different ways to address a fluids problem. First analysing a stationary control volume and applying conservation of mass equations to the borders. Secondly selecting a moving control volume of fixed mass and applying conservation of momentum equations to the borders. Thirdly selecting an infinitely small particle and analyzing the fluid that passes through it. Or finally selecting an infinitely small particle and analyzing its path through the fluid. This allows for the analysis of any problem from a variety of different perspectives.
\subsection{Optimal Design of an Archimedes screw}
As the device that we are manufacturing depends upon a screw similar to the original archimedes screw it was important that we analyze the methods used to optimize the design of one of these screws.\cite{Rorres00} This introduced the idea of breaking a design down into dimensionless variables that dictated the ratio of the size of components and reduced the amount of unknown variables. These dimensionless numbers were then varied and efficiencies were calculated providing the final, most efficient design ratios.
\subsection{Tom's Effect}
Fish are covered with mucus primarily to cover wounds and promote healing, however this mucus also causes Tom’s Effect. This is when a high molecular weight polymer is released into a fluid stream it makes the fluid remain laminar for longer because the polymers line up with the streamlines in the fluid. This requires the fluid to use more energy to break the streamlines and become turbulent. With fish this allows them to travel through the water with significantly less drag than they otherwise would have had to overcome. The could be applied to any underwater vessel to provide increased fuel efficiency and stealth.
\subsection{WHOI Acoustic Modem}
 Underwater communications have been a long standing challenge because radio waves attenuate rapidly in water. A possible method for communication between several underwater vehicles or even between a vehicle and its home base is an acoustic modem. These work by emitting modulated sound waves that carry data through the water. Due to the high speed of sound through water these methods of communication become practical. 
\subsection{CLT Propellers}
\begin{figure}[h]
\centering
\includegraphics[width=5cm]{"CLT"}
\caption{Contacted and Loaded Tip Propellers}
\end{figure}
Fundamentally the goal of the Contracted and Loaded Tip (CLT) propeller is to improve open water efficiency.  This means that the tip on the propeller reduces the velocities of water entering the propeller disk which, in turn, reduces the hydrodynamic pitch angle.  This reduction of hydrodynamic pitch angle and induced velocities results in many advantages.  To list a few, CLT propellers achieve higher top speeds, greater thrust, smaller optimum propeller diameter, better maneuverability, inhibits cavitation and tip vortices (resulting in less noise, less vibrations, lower pressure pulses, and lower area ratio
\subsection{Smart Duct}
\begin{figure}[h]
\centering
\includegraphics[width=5cm]{"Smart Duct"}
\caption{Thrust Vectoring Smart Duct}
\end{figure}
A Smart Duct is a deformable shroud that changes the direction of flow of the propeller wash to provide a direct steering force to the vehicle.The duct itself is an electrically actuated structure that is covered by a flexible hydrodynamically smooth sheathing whose primary movers are a set of high strength Nickel-Titanium SMA actuator cables.  Shape memory alloys (SMA) make this deformable Smart Duct a reality. Testing has proven flow turning angles of up to 15 degrees at thrust levels of operational submarines is possible with this technology.  These results may directly affect the design of future marine vehicles by reducing (and possibly eliminating) the use of control surfaces for maneuvering.
\subsection{Vortex Generators}
\begin{figure}[h]
\centering
\includegraphics[width=5cm]{"Vortex Generators"}
\caption{Vortex Generators on the Wing of a Fighter Jet}
\end{figure}
For relatively blunt objects like a sphere or wing on a plane the overall drag decreases when the boundary layer becomes turbulent because turbulent flow allows the boundary layer to follow the surface closer which decreases the overall wake region.  The larger this wake region is the more you see chaotic flow separation and adverse pressure gradients that can be catastrophic on aircraft because the flow separation can cause them to stall.Vortex generators can be found on the wings of aircraft and even on some high performance cars.  With vortex generators there is an exchange between high energy momentum and lower energy momentum by tripping laminar flow into turbulent which allows the boundary layer to remain attached over a greater length of the wing chord or car profile which results in a thinner wake region and smaller adverse pressure gradient on the rear of the object which lowers the pressure drag.  This allows for many benefits like lowering the stall speed, improving stability and control during maneuvering, and decreasing the turning radius.
\subsection{Wanda II}
\begin{figure}[h]
\centering
\includegraphics[width=5cm]{"Wanda II"}
\caption{Wrasse-inspired Agile, Near-shore, Deformable-fin Automation}
\end{figure}
Researchers have turned to nature for inspiration trying to model a UUV that uses flapping fins to maneuver through difficult underwater environments.  Many fish species use articulation of the pectoral fins to produce appropriate forces and moments propel themselves through the water and to react to dynamic changes in flow, physical obstacles, and wave forces near the shore. A four-fin UUV named WANDA-II (Wrasse-inspired Agile Near-shore Deformable-fin Automaton) is the 2nd generation of this alternative propulsion UUV. These fins are capable of producing thrust vectors in multiple directions through changes in curvature and stroke angle. All this is in the attempt to replicate the high level of controllability that fish species have near shore and in shallow water environments. A four-fin UUV could be deployed in a variety of missions including harbor monitoring and protection, hull inspection, and covert shallow water operations. 
\subsection{Control Surface Basics}
\begin{figure}[h]
\centering
\includegraphics[width=5cm]{"Pitch Roll Yaw"}
\caption{Representation of Euler Angles}
\end{figure}
Control Surfaces are moveable surfaces on wings that allow for maneuverability for both air and marine vehicles. Typically when control surfaces deflect they change the trailing edge of the wing which in turn changes the angle of attack. Angle of attack is the angle of between the wing chord (leading edge to trailing edge) to the Relative Airflow or free air flow (RAF). The amount of lift a wing produces is more a function of the angle of attack rather than the shape of the airfoil (cross section of the wing).The primary control surfaces on an airplane are ailerons (which controls roll/pitch), elevators (which controls pitch), and the rudder (which controls yaw).
\subsection{Materials and Stealth}

“…five major underwater vehicle requirements.  The two key issues are technical feasibility and stealth.  The third important issue is survivability.  Fourth, and equally important is the need to successfully deliver a payload.  And finally, all of these attributes regardless of importance, must be considered within a framework of cost.” (Submarine Technology for the 21st Century).\par
\begin{figure}[h]
\centering
\includegraphics[width=5cm]{"Submarine Stealth"}
\caption{Sonar Detection of a Submarine}
\end{figure}
Research was done into submarine hull materials and what we could learn from them for our project within the framework of parameters such as weight/speed, strength/depth capability, stiffness, corrosion resistance, magnetic signature, machinability/formability, fatigue resistance, thermal properties, electrical properties, and acoustic properties.  Research was conducted into steel, titanium, aluminum, composites (Carbon Fiber-, Titanium-, and Glass Fiber-Reinforced Polymers), and general polymers.\par
\par Stealth was investigated with regards to passive control (active control was not looked into owing to the impracticality with regards to the scope/scale of our project).  Passive control research centered firstly on the effect of polymeric bodies and polymer secretion on the hydrodynamic boundary layer, skin friction, wake field, general turbulence, vortices, and hydrodynamic noise generation.  The next direction for passive control was towards acoustic stealth via the use of anechoic materials (with sublayers for absorbing sonar waves, damping and decoupling of internal signal waves, and transmission layers for areas with intentional signal emission/transmission) and machinery “raft” beds (for vibration and noise damping), including attenuation ranges, elastomeric viscoelastic mechanical models, porosity effects (on attenuation, damping, and chemical stability), the correlation of relaxation modulus with temperature, signal frequency, and thermal transitions, and chemical stability (hydrolytic stability and water absorption).
\subsection{Funding}
Research was conducting with regards to sources of project financial support.  Options looked into included National Science Foundation (NSF) and Department of Defense (DoD) grants, innovation competition awards, (financial, material, product, and service) sponsorships (and the possible development of a sponsorship brochure) and donations, and online crowdsourcing.  The final decision was to use crowdsourcing for the majority of the fundraising and to look for sponsors and donors whenever possible.  Specific companies further researched for sponsorship, donation, or discount included Proto Labs, Rapid Machining, Quick Parts, Solid Concepts, and the Cal Poly Pomona Southern California Engineering Technologists Association (SCETA).\par
Websites looked into for hosting the crowdsourcing campaign included KickStarter, Indiegogo, RocketHub, and GoFundMe.  Aspects considered included campaign type [Flexible (keep what you raise) vs. Fixed (all or nothing)], timeline, campaign durations, processing/collection timelines and fees, donor rewards, website demographics (aimed at technical or artistic people), website traffic, and connectivity with social media.\par
At this time, the team also created social media accounts with Facebook, Instagram, and Twitter in order to help spread awareness of both the fundraising campaign and the website (which was created through WordPress.com after consideration and research into the same basic considerations as for the fundraising campaign, with the addition of website cost, domain ownership, and customization considerations).
\subsection{Testing}
Research was also conducted into both industry standard marine vehicle testing (mostly the use of tow tanks and the specific parameters often focused on with industry testing, i.e. the Admiralty Coefficient) and testing that we could perform. \par
\begin{figure}[h]
\centering
\includegraphics[width=5cm]{"Test Tank"}
\caption{Underside of a Boat in a Tow Tank}
\end{figure}
The first future test highlighted for our project included powered movement testing with stabilizing surfaces (for cancelling out torque transfer and rotation transmitted to the outer body) and a distributed weight belt (for inducing neutral buoyancy) with the parameters of interest of water speed, thrust, and flow field visualization.  Other test ideas consisted of battery run time tests, sealing tests, and small scale concept tests (i.e. using a cut open 2L bottle, cone to fit into the bottle, and swirling motion to determine the most efficient exit flow type through measuring drain time and waterproofing small DC motors and connecting them to vane shells of various geometries inside submerged tubes).
\subsection{Thrust Augmentation}
Research was conducted into Zone I (hull and pre-shaft inlet sections), Zone 2 (fluid working section), and Zone III (outlet section) thrust augmentation devices, their implementation, and how they influence propulsive efficiency, flow separation, turbulence, propulsor swirl (pre-swirl and post-swirl) (and thus flow equalization), cross flow minimization (and thus bilge vortices), effective thrust, shaft vibrations, wake field magnitude and distribution, open water efficiency, energy recovery, propeller hub vortices, conversion of rotational kinetic energy to directional flow, hydrodynamic lift, and lift-drag ratio.  Devices researched include:
\begin{itemize}
\item Zone I: Wake equalizing ducts, asymmetric sterns, Grothues spoilers, stern tunnels, semi- or partial ducts, reaction fins, Mitsui Integrated Ducted Propulsion Units, and Hitachi Zosen Nozzles.
\item Zone II: Increased diameter-low RPM propulsors, Grim Vane Wheels, propellers with end-plates, CLT propulsors, and propeller cone fins.
\item Zone III: Rudder-Bulb fin systems and additional thrusting fins.
\end{itemize}
\begin{figure}[h]
\centering
\includegraphics[width=5cm]{"Thrust Augmentation"}
\caption{Thrust Augmentation on the Hull of a Ship}
\end{figure}
\indent Research was also conducted into the effect of combination of devices with the conclusion that most combinations were not possible due to one device removing the regime utilized by the other, while a select few combinations led to good benefits.
\subsection{Owl Stealth}
\begin{figure}[h]
\centering
\includegraphics[width=8cm]{"Owl Stealth"}
\caption{Wing Features on an Owl that Provide Stealthy Flight}
\end{figure}
Research was conducted into the mechanisms by which owls derive their acoustic stealth, with the goal of better understanding the turbulent eddies and their amplification/scattering from the trailing edges of wings (natural or man-made) and how biomimicry might be able to change the way our control surfaces might affect our stealth (acoustic or flow signature).  Research centered on the profiles, material properties (most importantly stiffness), geometry, and surface roughness of the leading edge, mid-wing section, and trailing edge.\par
It was determined that the trailing edge is the dominant noise source on wings and that owl wing stealth was mainly the result of use of leading and trailing edge tubercles and flexible, porous trailing edge material.  Tubercles were to be a future feature of our vehicle's control surfaces, time permitting.
\subsection{Submarine Hydrostatics}
\begin{figure}[h]
\centering
\includegraphics[width=8cm]{"Submarine Hydrostatics"}
\caption{Center of Buoyancy and Center of Mass of a Submarine}
\end{figure}
Research was conducted into ship and submarine hydrostatics in the surfaced, diving/surfacing, and submerged states.  Key factors considered included state of buoyancy, reserve of buoyancy, trim, configuration and use of main ballast tanks (internal and external), free flood spaces, margin ballasts, variable ballasts, superstructures, flexible bag ballasts with flotation collars, freeboard, and a pressure hull.\par
\indent Methodologies for buoyancy calculation were researched heavily for both intact and damaged vessel states, surface (static), surfacing/diving (transition), and submerged (static) states, conditions of flotation, and disturbance response for small and large values of sway, yaw, surge, heave, and (the most complex case) heel/roll.\par
\indent Some specific takeaways included the final decision with regards to buoyancy control system and flood/vent hole design (i.e. location and use of a grill if requiring a large hole in order to disrupt noise, vibration, and drag caused by the flood/vent hole(s)).
\subsection{Submarine Design Process}
\begin{figure}[h]
\centering
\includegraphics[width=5cm]{"Submarine Design"}
\caption{Design Considerations for Underwater Vehicles}
\end{figure}
Research was conducted into the submarine design process as presented in Concepts in Submarine Design, 2s (Burhcer).  This design process is shown below, with the accelerated modifications for our time-sensitive project shown in red.\par
Note: Bolded text in the following paragraphs highlights the applicability to our project. Specific topics focused on included  role designation \textbf{(ISR)} [and subsequent subrole \textbf{(perform ISR in a covert manner within a variety of environments while retaining high stealth, maneuverability, elevated speeds, and sufficient battery life)]}, development of operational requirements \textbf{(specifics of speed, maneuverability, battery life, stealth characteristics, etc. requirements)}, concept studies (with key parameters of size, cost, payload, performance), feasibility studies (and system material requirements) \textbf{(development of weight, space, and power budgets for various sub-systems in design)}, \textit{design for build} (detailed systems, subsystems, arrangements, configurations, specifications, etc. \textbf{(management of previously developed budgetary allocations with deliverables relating to structures, arrangements, hydrodynamics, subsystems, hydrostatics (both surface and submerged), etc.)}, and production design \textbf{(final vehicle design)}.
\subsection{Buoyancy Control}
A research article contained information related to buoyancy control of a semiautonomous underwater vehicle and design orientation. The orientation of the design determined what the application of the vehicle. For example, if the design were to be horizontal resembling a manta ray, its use would be for towing and exploration at lower depths. However, due to this orientation it has a higher chance of getting stuck in densely populated sea vegetation or between underwater rocks. If the orientation were to be vertical, resembling a fish, its use would be for a remote controlled mode and towing as well. The vertical orientation makes it easier to maneuver in densely populated sea vegetation. The buoyancy control mechanism contained two compressed air tanks connected to two separate balloons. The air would be released into the balloon when the vehicle needed to raise and the air would be released the vehicle needed to be lowered.
\subsection{Underwater Coatings}
Cathodic protection is used in a vary of underwater applications. Cathodic Protection (CP) is a technique used to control the corrosion of a metal surface by making it the cathode of an electrochemical cell. A simple method of protection connects the metal to be protected to a more easily corroded "sacrificial metal" to act as the anode. The sacrificial metal then corrodes instead of the protected metal. The sacrificial metal that has been corroded provides a protective surface to prevent the base material from being corroded. Alocit by the A\&E group is an coating that is used for underwater applications. It is one of three coatings that meet the specs of the US Army Corps of Engineers.
\subsection{Currents}
\begin{figure}[h]
\centering
\includegraphics[width=5cm]{"Currents"}
\caption{Currents and Overall Fluid Transport}
\end{figure}
Currents are broken down into two types, surface and deep water. Deep water currents usually occur more than 400 meter (~1300 feet) under the surface of the water. The water movements are caused by differences in water density known as Thermohaline circulation. Deep water currents are much slower than surface currents but they move much more water due to water density increase deeper in the sea. Surface currents are due to the Coriolis Effect. The Coriolis Effect occurs due to the rotation of the earth, the circulating air is deflected resulting in curved paths. The wind’s curved path drags on the water’s surface, causing it to move in the direction the wind is blowing. The Ekman Spiral is a consequence of the Coriolis Effect.
\section{TeamUV.org}
Establishing TeamUV.org was the obvious next step after a Summer filled with professional growth and academic exploration.  Not only was the website about the current progress of the team’s senior project but also about inspiring interest in STEM and facilitating monetary support.  Topics covered range from fluid dynamics, robotics, and biomimicry to shining light on what engineering is and how it impacts the world.  Inspired by websites often visited by Team UV members, posts were uploaded various times throughout the week to keep followers intrigued without feeling overwhelmed.  Well Reads, Presentations, and Open Minds were set to be posted on Tuesdays, Thursdays, and Sundays, respectively.  Each post was written by a team member according to a set schedule set for months in advance.  A list of website postings can be found in Appendix .  Team UV also made sure that the website was linked with companion Facebook, Twitter, Instagram, and GoFundMe accounts.\par
\indent Team UV.org went live on July 31, 2014 with a welcome post about the site, the project, and ways to help the team’s fundraising campain.  The site instantly created buzz pulling in 122 views and 38 visitors the first month.  After gaining a steady following from all around the world including 87 countries, TeamUV.org achieved a best month of 670 views and 518 unsubscribed visitors in January 2015.  To date, the site has gained a steady, recurring base of 84 subscribed followers that actively visit and participate in leaving comments for the team.\par
\indent Aside from weekly posts, visitors to the site can also find team member biographies giving background about personal interests, academic experiences, and professional work.  Those looking to sponsor the team’s efforts with donations or equipment are also able to give directly on the site, follow a  link to the Team UV GoFundMe site, or contact the them directly.  

\section{Summer Testing}
Along with delving into Summer research pertaining to many aspects of engineering design, Team UV also performed a couple of tests.  Another trip to Sailboat Cove was made to test the benefits of attaching control surfaces to Sheila-D.  Spring Quarter testing resulted in the formation of a free exit jet but minimal displacement of the vehicle.  Due to stacking inefficiencies with material availability, available manufacturing methods, and unaddressed torque transfer issues, Sheila-D was in need of support.  Scrap square pieces of aluminum flashing were roughly attached to the sides of the propulsor which controlled the overturning moments caused by torque transfer in the device.  This resulted in linear travel along the dock, verifying the projects potential.\par
Testing the importance of the exit cone was also performed during the summer session.  2-liter soda bottles were modified to resemble the flow through Sheila-D.  Water was passed through the model’s nozzle. with and without an exit cone, and discharge times were recorded.  The use of the exit cone was verified by the faster exit times of the cone-equipped soda bottle.  Tests were also performed as to the effects of pre-swirl into the nozzle caused by the vane shell.  Tests using modified soda bottles showed promising results as the water look to discharge out the nozzle faster with pre-swirl but were eventually ruled inconclusive.  Controlling the swirl of the exiting water was difficult to accomplish giving the tests poor repeatability.\\
\chapter{Phase III}
\section{Lessons Learned}
\section{Subsystem Decomposition}
\subsection{Propulsor}
\subsubsection{Control Volumes}
\subsubsection{MatLab Code}
The mathematical model rendered through the above control volume analysis still contained many unknowns. With the above equations the relationship between these unknowns could be determined, however finding the optimal value of each of the 5 geometrical dimensions and the one dynamic variable would have required exhaustive iterations and guess work. To facilitate the design process and ensure optimal design a program was written in the MatLab coding environment that would cycle through ranges of each of these variables, creating every possible combination of designs and rank them on an efficiency. This would allow for the most efficient design to be selected.\par
The majority of the variables in question described the geometry of the propulsor. To reduce the complexity of the design some variable were designated before the program began. The length of the nozzle was fixed to 5 inches, providing a reference size so that the most efficient design would still have a reasonable package size. The number of vanes was set to 5, in previous iterations of the program a value of 5 vanes commonly populated the highest efficiency and therefore it was set, freeing up processing time for more volatile variables. The outer radius, inner radius, length, angle of the vane helix, and nozzle radius were allowed to vary. Each variable had to have a range to vary within because without a finite range the program would loop infinitely. The maximum outer radius was set to 6 inches, the minimum inner radius was set to 1 inch, length ranged between 12 and 36 inches, the vane angle varied between 30 and 60 degrees, the minimum nozzle radius was .001 inches.\par
%INSERT TABLE WITH INITIAL RANGES AND VARIABLES%
The dynamics of the system also had to be coded into the program. The angular velocity was the only parameter allowed to vary from a low speed of 50 rad/sec to 200 rad/sec. A pump efficiency of .75 was used to account for losses in energy transfer from the vanes to the fluid itself, the surface roughness of PVC was used because it most accurately modeled the material we anticipated the vane shell to be made of. A drag coefficient of 1.2 was used as a conservative estimate, the real drag will be much less because the outer surface will be streamlined.\par
 %INSERT TABLE OF VARIABLES WITH REFERENCES%
Not every variable needed a range to vary within because certain relationships were established. This allowed for more accurate results while still minimizing the amount of processing time. The minimum outer radius was limited to be at least .25in greater than the inner radius to allow for a practical vane height. The nozzle maximum value was limited so that the exit area would never exceed the area of the annular region between the inner cylinder and the outer cylinder. This was done to ensure that the the nozzle would be converging and provide increased thrust rather than the diverging nozzle that would slow the fluid and reduce the overall thrust.  \par
With these relationships and ranges established the order of the nested loops needed to be established. The loops were ordered so that the more depended on the variable being changed the farther out of the nest it was placed. This led to the following order: angular velocity, inner radius, outer radius, nozzle radius, length, then vane angle. By nesting the loops in this order all of the relationships between the different range limits could be satisfied and once in the center of the loops one complete set of design variables was established.\par
In the center of the nested loops two finite difference loops were used to assist in the solving of the mathematical models derived from the control volume analysis. The first finite difference loop calculated the friction factor used to find the losses along the length of the vane shell. The second finite difference loop was used to find the friction factor for the losses through the nozzle. These losses were used with the established design variables to find the overall thrust, the pressure at the end of the vane shell, the speed of the device at steady state, and the speed of the free jet leaving the end of the nozzle.\par
These results from the fluid analysis were used to determine the efficiency of the device. The energy added to the fluid increased the velocity up to the end of the vane cylinder. A 100\% efficient device would move at the speed of the fluid at this point. Essentially a fully efficient device would accelerate until it could no longer add energy to the fluid. This means that the efficiency of the device could be calculated by taking the steady state speed of the device and dividing it by the steady state speed of the fluid after the vane shell. $\eta=\frac{V_1}{V_2}$. These calculations were done for all the possible combinations of design parameters within the given ranges resulting in thousands of possible designs each with their own efficiency.\par
\subsubsection{Excel Narrowing of Results}
All of the designs were exported in a comma separated value format to be manipulated in Excel. First all efficiency values less than or equal to zero were removed as poor designs, then each variable was graphed with respect to efficiency. Due to the cyclic nature of the analysis data was plotted as clusters around certain values. Using these plots the clusters of data with low efficiency could be eliminated. Certain geometrical values were limited by realistic manufacturing constraints. This means that very small clearances between the inner radius and outer radius were removed along with  excessively small nozzle radi. By narrowing these results in this fashion final design values were arrived upon with a resulting efficiency of between 77.8\% and 80.7\% for a range of angular velocities. The final geometry was.
%INSERT TABLE WITH FINAL GEOMETRY%
\subsubsection{CFD}
\subsubsection{Manufacturing}
\begin{thebibliography}{9}
\bibitem{Rorres00}
Rorres, Chris. "The Turn of the Screw: Optimal Design of an Archimedes Screw." Journal of Hydraulic Engineering (2000): 72. Print.
\end{thebibliography}
\end{document}