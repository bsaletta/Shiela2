\documentclass{report}
\usepackage{fullpage}
\usepackage{graphicx}
\graphicspath{{figures/}}
\title{Design of An Innovative, Highly Maneuverable, Stealthy Unmanned Underwater Vehicle with ISR Capabilities }
\author{Andrew Blancarte\\ Ketton James\\ Brian Martin\\ Abraham Paucar\\ Ben Saletta}
\date{\today}

\begin{document}
\maketitle
\tableofcontents
\flushleft
\chapter{Phase I: Propulsor Development, SHEILA-D}
\section{Overview}
SHEILA-D stands for Subaquatic Hydrodynamicaly propelled Explorer Implementation Los Angeles - Demonstrator. The driving force behind the development of this unmanned underwater vehicle is the innovative propulsion system. This system was the foundation of the first phase of development, a proof of concept for this screw based propulsion system. As this phase was a deliverable for a Machine Design Class there is a detailed overview of this phase included in appendix A. 
\section{Constraints}
As this initial design was part of a class project there were many constraints that came along with it. The primary constraint was time. The project was completed, from design to testing, in 35 days to meet the deadline imposed by the academic calender. There were also significant cost constraints because this phase was funded completely out of the personal funds of the members involved. The constraints eliminated several custom manufacturing techniques due to extensive lead times and costs. This led to the final construction consisting mainly of parts shaped from hardware store components.
\section{Design}
The initial design phase took the first 5 days in which the foundational fluid mechanics, the drive train and power calculations were completed in these days. Further redesign was included in with the manufacturing process when the aforementioned constraints restricted the construction of the device as designed.
\subsection{Fluids Calculations}
The fluids calculations were an important part of the initial design because of the roll they played in moving the device through the water. To complete these calculations the device was broken into two sections. First the cylinder holding the screw and second the cone used to accelerate the fluid. For the cylinder an assumption was made that a constant force would be imparted to the fluid along the surface of the screw blades. This constant force would act on the fluid mass between the vanes therefore, assuming a constant density, the vanes would provide a constant acceleration along this section. With a known length of the cylinder and this calculated acceleration the speed of the fluid exiting the cylinder was calculated. Conservation of momentum was then applied to the fluid as it passed through the final cone and nozzle giving the final thrust force of the propulsion system. These calculations were plugged into an Excel spreadsheet and various geometries were iterated through to find the optimal configuration. A more detailed walk through of the fluids analysis is included in appendix A.
\subsection{Drive Train}
The fluids calculations provided a value for the required torque and angular velocity. The drive train was designed to meet both of these criterion. The major difficulty with the drive train in this design was transmitting the torque from the motors to the external shell. To solve this problem and provide sufficient torque two motors were fitted with gearboxes then mated to the sun gear in a planetary gear set that transmitted the torque out to the rotating cylinder as seen in Figure 1.1. The planetary gear system was 3D printed to minimize cost and manufacturing time.
\begin{figure}[h]
\centering
\includegraphics[width=15cm]{"Section View"}
\caption{Drive Train Design Section View}
\end{figure}
\subsection{Power Supply}
These motors required a significant amount of current at a constant 12 volts. These requirements went into the power design calculations and searches for sufficient batteries. A single 12 volt lithium battery was selected due to its high current discharge, compact form, and high capacity characteristics. The design of the power supply was integrated into the design of the drive train due to the important relationship between the easily available motors and batteries for a reasonable operational time. The power supply had to fit between the two motors and is shown as a black box in Figure 1.1.
\section{Construction}
The construction of this device was completed in a very short time span using commercial off the shelf components. This involved PVC pipes of various diameters, cutting and thermoforming sheets of acrylic, and fastening components with traditional fasteners as well as epoxy. As the construction proceeded difficulties were discovered and the design was adjusted to compensate for these problems. One of the major difficulties was the manufacture of the helix screw. This was solved by thermoforming cut acrylic then filling gaps created in the vanes with waterproof tape. Other adjustments and quick fixes were applied through out the process. The construction was a challenge to complete within the time frame at an acceptably low cost.  
\section{Testing}
Once construction was completed the device needed to be tested to confirm that the propulsion system did indeed propel the device through the water. To do the testing properly the vehicle was taken to the Puddingstone reservoir at dawn and ran along the deck of a boat dock. It took a few tests to get the device operational and, while the vehicle could not be set off on its own at this stage, it pumped a significant amount of water though the body and out the cone and nozzle assembly. Many things were learned about the design during the testing session and made note of for future design. The major lessons were that the assumptions that were made in the fluid calculations were wrong and that without control surfaces the vehicle would reach an equilibrium rotation state where the outermost cylinder would rotate and counter the rotation of the center cylinder. 

\end{document}