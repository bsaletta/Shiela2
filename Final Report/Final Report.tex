\documentclass{report}
\usepackage{fullpage}
\usepackage{graphicx}
\graphicspath{{figures/}}
\title{Design of An Innovative, Highly Maneuverable, Stealthy Unmanned Underwater Vehicle with ISR Capabilities }
\author{Andrew Blancarte\\ Ketton James\\ Brian Martin\\ Abraham Paucar\\ Ben Saletta}
\date{\today}

\begin{document}
\maketitle
\tableofcontents
\flushleft
\chapter{Phase I: Propulsor Development, SHEILA-D}
\section{Overview}
SHEILA-D stands for Subaquatic Hydrodynamicaly propelled Explorer Implementation Los Angeles - Demonstrator. The driving force behind the development of this unmanned underwater vehicle is the innovative propulsion system. This system was the foundation of the first phase of development, a proof of concept for this screw based propulsion system. As this phase was a deliverable for a Machine Design Class there is a detailed overview of this phase included in appendix A. 
\section{Constraints}
As this initial design was part of a class project there were many constraints that came along with it. The primary constraint was time. The project was completed, from design to testing, in 35 days to meet the deadline imposed by the academic calender. There were also significant cost constraints because this phase was funded completely out of the personal funds of the members involved. The constraints eliminated several custom manufacturing techniques due to extensive lead times and costs. This led to the final construction consisting mainly of parts shaped from hardware store components.
\section{Design}
The initial design phase took the first 5 days in which the foundational fluid mechanics, the drive train and power calculations were completed in these days. Further redesign was included in with the manufacturing process when the aforementioned constraints restricted the construction of the device as designed.
\subsection{Fluids Calculations}
The fluids calculations were an important part of the initial design because of the roll they played in moving the device through the water. To complete these calculations the device was broken into two sections. First the cylinder holding the screw and second the cone used to accelerate the fluid. For the cylinder an assumption was made that a constant force would be imparted to the fluid along the surface of the screw blades. This constant force would act on the fluid mass between the vanes therefore, assuming a constant density, the vanes would provide a constant acceleration along this section. With a known length of the cylinder and this calculated acceleration the speed of the fluid exiting the cylinder was calculated. Conservation of momentum was then applied to the fluid as it passed through the final cone and nozzle giving the final thrust force of the propulsion system. These calculations were plugged into an Excel spreadsheet and various geometries were iterated through to find the optimal configuration. A more detailed walk through of the fluids analysis is included in appendix A.
\subsection{Drive Train}
The fluids calculations provided a value for the required torque and angular velocity. The drive train was designed to meet both of these criterion. The major difficulty with the drive train in this design was transmitting the torque from the motors to the external shell. To solve this problem and provide sufficient torque two motors were fitted with gearboxes then mated to the sun gear in a planetary gear set that transmitted the torque out to the rotating cylinder as seen in Figure 1.1. The planetary gear system was 3D printed to minimize cost and manufacturing time.
\begin{figure}[h]
\centering
\includegraphics[width=15cm]{"Section View"}
\caption{Drive Train Design Section View}
\end{figure}
\subsection{Power Supply}
These motors required a significant amount of current at a constant 12 volts. These requirements went into the power design calculations and searches for sufficient batteries. A single 12 volt lithium battery was selected due to its high current discharge, compact form, and high capacity characteristics. The design of the power supply was integrated into the design of the drive train due to the important relationship between the easily available motors and batteries for a reasonable operational time. The power supply had to fit between the two motors and is shown as a black box in Figure 1.1.
\section{Construction}
The construction of this device was completed in a very short time span using commercial off the shelf components. This involved PVC pipes of various diameters, cutting and thermoforming sheets of acrylic, and fastening components with traditional fasteners as well as epoxy. As the construction proceeded difficulties were discovered and the design was adjusted to compensate for these problems. One of the major difficulties was the manufacture of the helix screw. This was solved by thermoforming cut acrylic then filling gaps created in the vanes with waterproof tape. Other adjustments and quick fixes were applied through out the process. The construction was a challenge to complete within the time frame at an acceptably low cost.  
\section{Testing}
Once construction was completed the device needed to be tested to confirm that the propulsion system did indeed propel the device through the water. To do the testing properly the vehicle was taken to the Puddingstone reservoir at dawn and ran along the deck of a boat dock. It took a few tests to get the device operational and, while the vehicle could not be set off on its own at this stage, it pumped a significant amount of water though the body and out the cone and nozzle assembly. Many things were learned about the design during the testing session and made note of for future design. The major lessons were that the assumptions that were made in the fluid calculations were wrong and that without control surfaces the vehicle would reach an equilibrium rotation state where the outermost cylinder would rotate and counter the rotation of the center cylinder. 
\chapter{Phase II: Summer Research and Testing}
During the summer of 2014 the group members of Team UV did not entirely part ways for three months. The group decided that it would be in the best interest of the group and design itself to meet twice a month where each member would give presentation on standard material along with research they have found during the two week period between meetings. These meetings started in June of 2014 and ended when the 2014 Fall period started, September 2014.  Each of the meetings held were roughly five to seven hours long. The format of the presentations were standardized however each member provided their own research and information. Presentation format can be found in appendix . In the summer of 2014 Team UV spent roughly 240 hours as a group towards the senior project itself. That number includes research done outside of the meeting along with the meetings themselves. The PowerPoint presentations contained three sections.
\section{Open Mind}
The group was given a situation/problem that has or could happen in the world we live in today and each member had to give insight or propose a solution to the problem/situation. Each member were asked to provide their solution along with reasoning  as to why their solution would be efficient. Since Team UV are composed of mechanical engineering undergraduates, the solutions based on the prompts were required to use some aspect of engineering. For example, one open mind prompt addressed the issue that many people in developing countries lack the access to power. The group was asked if they were tasked with developing an inexpensive “do it yourself” power generation system for use in a 3rd world country, what kind of engineering considerations might you take into account. The idea behind the open mind prompt was to keep the group members mind engaged during the break and keep that engineering mentality sparked. Also, this prompt was to have the members think outside the box and to aid in developing creative/innovative ideas for use in the senior project design.
\section{Well-Read}
Members were asked to look into articles/publications and present something they found to be interesting, innovative, advancements, etc. The articles were not limited to strictly the engineering field however it was recommended.  Articles presented from the group ranged from next generation wind turbines to theory behind the process of suction eating of fish. The idea of the well-read was to give insight to the group of remarkable advancements in science and technology throughout the world.
\section{Presentation}
The presentation was an open ended section which allowed for each member to present some research they did during the two week period between meetings. The research was required to aid in the design of the senior project in areas such as theory, analysis, design, controls, manufacturing, and/or testing. Research in CLT propellers, computational fluid dynamics, Arduino coding, wake turbulence, underwater designs and coatings, power management, component selection for marine research, smart ducts, submarine design/research, and many others. 
\section{Summer Research Summaries}
\subsection{Shark Skin}
Sharks move very efficiently thanks to the characteristics of their skin.  Anything but smooth, shark skin has individual scales, called dermal denticles, with narrow passages that reduce friction drag and accelerate the flow along its length.  These scales also flex and realign to reduce biofouling which can negatively affect flow over the body.
\subsection{Vantablack}
Surrey NanoSystems has engineered a new super-black material called Vantablack.  This material composed of “Vertically Aligned Carbon Nanotube Arrays” absorbs 99.965% of incident light.  Photons are allowed into the material and are then blocked and trapped from leaving.  This material is already in high demand for space and stealth applications. 
\subsection{Cephalopod Skin}
Two teams of researchers from Rice University and MIT were tasked with developing material which could replicate the camouflaging abilities of cephalopods.  The class of mollusks which include squid, possess skin that can manipulate its own color and texture.  Rice University engineering a rigid aluminum nanorod display panel which displays an intense color spectrum.  MIT, on the other hand, created a flexible display that can change color and texture but with limited color spectrum.
\subsection{Waterproofing Sensors}
Using sensors underwater with a strict budget takes a bit of creativity.  Two popular methods used by hobbyists and professionals alike are: Potting and Conformal Coating.  Potting is the method of filling an electronic assembly with a solid or gelatinous compound such as a thermosetting plastic or silicone.  This blocks water and increases shock resistance.  Conformal Coating offers the same benefits of potting but is lighter and non-permanent which allows for reworking.
\subsection{Arduino}
Microprocessing is made accessible to all thanks to companies like Arduino.  Boards housing everything from input and output terminals to a microprocessor can link the electrical world to the physical world.  Users start out controlling the lighting sequencing of LED’s to eventually controlling more advanced mechanical systems.  Topics presented to the team include LED control, reading data from sensors, and ways to expand the factory limitations of Arduino boards.  
\subsection{CFD and Python}
Computational Fluid Dynamics, CFD, is a powerful tool for the analysis of fluid behavior. The computational techniques used to break down the navier-stokes equations have applications that stretch beyond the realm of fluid dynamics and allow the modeling of many non linear mathematical phenomena within a computational environment. This presentation was an introduction to the ideas behind CFD and the way that an analysis could be written using the Python coding language.
\subsection{Physical Understanding of Fluids Equations and their Application to CFD}
To apply any of the fundamental fluids equations to a device a physical understanding of the meaning behind the mathematics is crucial. This presentation addressed four different ways to address a fluids problem. First analysing a stationary control volume and applying conservation of mass equations to the borders. Secondly selecting a moving control volume of fixed mass and applying conservation of momentum equations to the borders. Thirdly selecting an infinitely small particle and analyzing the fluid that passes through it. Or finally selecting an infinitely small particle and analyzing its path through the fluid. This allows for the analysis of any problem from a variety of different perspectives.
\subsection{Optimal Design of an Archimedes screw}
As the device that we are manufacturing depends upon a screw similar to the original archimedes screw it was important that we analyze the methods used to optimize the design of one of these screws. [1] This introduced the idea of breaking a design down into dimensionless variables that dictated the ratio of the size of components and reduced the amount of unknown variables. These dimensionless numbers were then varied and efficiencies were calculated providing the final, most efficient design ratios.

\section{TeamUV.org}
Establishing TeamUV.org was the obvious next step after a Summer filled with professional growth and academic exploration.  Not only was the website about the current progress of the team’s senior project but also about inspiring interest in STEM and facilitating monetary support.  Topics covered range from fluid dynamics, robotics, and biomimicry to shining light on what engineering is and how it impacts the world.  Inspired by websites often visited by Team UV members, posts were uploaded various times throughout the week to keep followers intrigued without feeling overwhelmed.  Well Reads, Presentations, and Open Minds were set to be posted on Tuesdays, Thursdays, and Sundays, respectively.  Each post was written by a team member according to a set schedule set for months in advance.  A list of website postings can be found in Appendix .  Team UV also made sure that the website was linked with companion Facebook, Twitter, Instagram, and GoFundMe accounts.\\
\indent Team UV.org went live on July 31, 2014 with a welcome post about the site, the project, and ways to help the team’s fundraising campain.  The site instantly created buzz pulling in 122 views and 38 visitors the first month.  After gaining a steady following from all around the world including 87 countries, TeamUV.org achieved a best month of 670 views and 518 unsubscribed visitors in January 2015.  To date, the site has gained a steady, recurring base of 84 subscribed followers that actively visit and participate in leaving comments for the team.\\
\indent Aside from weekly posts, visitors to the site can also find team member biographies giving background about personal interests, academic experiences, and professional work.  Those looking to sponsor the team’s efforts with donations or equipment are also able to give directly on the site, follow a  link to the Team UV GoFundMe site, or contact the them directly.  

\section{Summer Testing}
Along with delving into Summer research pertaining to many aspects of engineering design, Team UV also performed a couple of tests.  Another trip to Sailboat Cove was made to test the benefits of attaching control surfaces to Sheila-D.  Spring Quarter testing resulted in the formation of a free exit jet but minimal displacement of the vehicle.  Due to stacking inefficiencies with material availability, available manufacturing methods, and unaddressed torque transfer issues, Sheila-D was in need of support.  Scrap square pieces of aluminum flashing were roughly attached to the sides of the propulsor which controlled the overturning moments caused by torque transfer in the device.  This resulted in linear travel along the dock, verifying the projects potential.\\
\indent Testing the importance of the exit cone was also performed during the summer session.  2-liter soda bottles were modified to resemble the flow through Sheila-D.  Water was passed through the model’s nozzle. with and without an exit cone, and discharge times were recorded.  The use of the exit cone was verified by the faster exit times of the cone-equipped soda bottle.  Tests were also performed as to the effects of pre-swirl into the nozzle caused by the vane shell.  Tests using modified soda bottles showed promising results as the water look to discharge out the nozzle faster with pre-swirl but were eventually ruled inconclusive.  Controlling the swirl of the exiting water was difficult to accomplish giving the tests poor repeatability.\\



\end{document}